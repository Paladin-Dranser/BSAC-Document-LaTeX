%%% Пункт 3.1 — фармат А4 (210x297 мм);
%%%           — памер асноўнага шрыфту 12 пунктаў;
\documentclass[a4paper,12pt]{extarticle} % extarticle — падтрымка шрыфтаў з памерам 8pt, 9pt, 10pt, 11pt, 12pt, 14pt, 17pt, 20pt

%%% Пункт 3.1 — памер палёў старонкі (малюнак 3.1, СТП 01-2017)
\usepackage{geometry}
    \geometry{left=30mm}
    \geometry{right=10mm}
    \geometry{top=20mm}
    \geometry{bottom=20mm}

%%% Пункт 3.1 — шрыфт Times New Roman
\usepackage{fontspec}
    \defaultfontfeatures{Renderer=Basic,Ligatures={TeX}}
    \setmainfont{Times New Roman}

%%% Падключэнне беларускай мовы
\usepackage[english, belarusian]{babel}

%%% Дазвол на кірыліцу ў формулах
\usepackage{mathtext}

%%% Для ўключэння pdf дакументаў у канчатковы файл
\usepackage{pdfpages}

%%% Пераносы ў словах са злучком
%%% Злучок у словах замяняецца на \hyph: апаратна\hyphпраграмны
%%% https://stackoverflow.com/questions/2193307/how-to-get-latex-to-hyphenate-a-word-that-contains-a-dash
\def\hyph{-\penalty0\hskip0pt\relax}

%%% мадыфікатары напісання
\usepackage{soul}

%%% спасылкі ў pdf
\usepackage{hyperref}
    \hypersetup{
        unicode=true,                           % кірыліца ў раздзелах pdf
        pdftitle={Дыпломны праект},             % загаловак
        pdfauthor={Буцько Антось Уладзіміравіч},% аўтар
        colorlinks=true,                        % каляровыя спасылкі
        linkcolor=red,                          % колер унутраных спасылак
        linktoc=all,                            % спасылка па загалоўку і нумару старонкі
    }

%%% Пункт 3.1 — інтэрвал паміж радкамі
%%%             мае складаць 1,5 машынапіснага інтэрвалу
\usepackage{setspace}
    \onehalfspacing

%%% «Разумная» коска
\usepackage{icomma}

%%% перанос знакаў у формулах (па Львоўскаму)
%%% https://www.coursera.org/learn/latex/
\newcommand*{\hm}[1]{#1\nobreak\discretionary{}
{\hbox{$\mathsurround=0pt #1$}}{}}

%%% подпісы
\usepackage{caption}
    %%% Пункт 3.3 — подпіс табліцы
    \captionsetup[table]{
        name=Табліца,
        labelsep=endash,                % значок " — "
        justification=raggedright,      % выраўноўваць па праваму боку
        singlelinecheck=false,          % не выраўноўваць па цэнтру, калі 1 радок
        skip=0pt,                       % паміж табліцай і назвай няма адлегласці
        font=onehalfspacing,            % 1,5 інтэрвал паміж радкамі
        belowskip=-0.2\baselineskip,    % "магія" для роўных прамежкаў зверху і знізу табліцы (падбіраў для прыблізнай аднолькавасці ў міліметрах)
    }
    %%% Пункт 3.5 — подпіс ілюстрацый
    \captionsetup[figure]{
        name=Малюнак,
        labelsep=endash,                % значок " — "
        justification=centering,        % выраўноўваць па цэнтру
        skip=1.5\baselineskip,          % пусты радок паміж малюнкам і назвай
        font=onehalfspacing,            % 1,5 інтэрвал паміж радкамі
        belowskip=-0.6\baselineskip     % "магія" для роўных прамежкаў зверху і знізу малюнка
    }

\usepackage{fancyhdr} % пакет для устаноўкі калонтытулаў
    %%% Стыль старонкі (раздзел)
    %%% Пункт 3.2 — нумарацыя старонак у ніжнім правым кутку
    %%%           — (малюнак 3.1, СТП 01-2017)
    \fancypagestyle{section}{
        \fancyhf{} % ачыстка стылю
        \renewcommand{\headrulewidth}{0pt} % прыбраць падзяляльную лінію ўверсе старонкі
        \renewcommand{\footrulewidth}{0pt} % прыбраць падзяляльную лінію ўнізе старонкі
        \rfoot{\thepage} % нумарацыя ў ніжнім правым кутку
    }

%%% Водступ першага абзаца ў section (адпаведна з астатнімі абзацамі)
\usepackage{indentfirst}

%%% адключэнне змянення прабелаў паміж словамі/сказамі
\frenchspacing

%%% Пункт 3.1 — велічыня водступу абзаца 1,25 см
\setlength{\parindent}{1.25cm}

%%% магчымасць адключаць пераносы слоў
\usepackage{hyphenat}

%%% Рэдагаваць выгляд загалоўкаў
\usepackage{titlesec}
%%% Пункт 3.1 — загалоўкі раздзелаў пішуцца вялікімі літарамі
%%%           — водступ абзаца
%%%           — паўтоўсты шрыфт Times New Roman 14 памеру
%%%           — перанос слоў не дапускаецца
%%%           — адлегласць паміж загалоўкам і тэкстам у 1 прабельны радок
%%%           — адлегласць паміж загалоўкам і падзагалоўкам у 1 прабельны радок
%%%           — дадатак Р
    \titleformat{\section}
        {\fontsize{14}{14} \bfseries \raggedright}
        {}{0pt}{\hspace{1.25cm}\thesection~\MakeUppercase} 
    \titlespacing{\section}{0pt}{0pt}{\baselineskip}
%%% Пункт 3.1 — паўтоўсты шрыфт Times New Roman з памерам асноўнага тэксту
%%%           — адлегласць паміж падзагалоўкам і тэкстам у 1 прабельны радок
%%%           — водступ абзаца
%%%           — дадатак Р
    \titleformat{\subsection}
        {\bfseries \raggedright}
        {}{0pt}{\hspace{1.25cm}\thesubsection~} 
    \titlespacing{\subsection}{0pt}{\baselineskip}{\baselineskip}
%%% Пункт 3.1 — нармальны шрыфт Times New Roman
%%%           — нумар падпункта паўтоўстым шрыфтам Times New Roman
%%%           — водступ абзаца
%%%           — загаловак пункта не вылучаецца паўтоўстым шрыфтам
%%%           — змест пункта пачынаецца з новага абзаца без прабельнага радка
    \titleformat{\subsubsection}
        {\fontsize{12}{12} \raggedright}
        {}{0pt}{\hspace{1.25cm}{\bfseries \thesubsubsection~}} 
    \titlespacing{\subsubsection}{0pt}{\baselineskip}{0pt}

%%% Змест
\usepackage{tocloft}
%%% Падпункт 2.2.6
%%% Дадатак И
    % змест не нумаруюць
    \tocloftpagestyle{empty}
    % адключэнне пераносу слоў
    \makeatletter
        \renewcommand{\@tocrmarg}{2.55em plus1fil}
    \makeatletter
    % шматкроп'е да нумару старонкі
    \cftsetpnumwidth{0pt}
    % адлегласць паміж нумарам раздзела і яго назвай
    \renewcommand{\cftsecnumwidth}{10pt}
    % шрыфт загалоўка раздзела
    \renewcommand{\cftsecfont}{\normalfont}
    % шрыфт нумара раздзела
    \renewcommand{\cftsecpagefont}{\normalfont}
    % шматкроп'е паміж загалоўкам і нумарам старонкі для раздзела
    \renewcommand{\cftsecleader}{\cftdotfill{\cftdotsep}}
    % адлегласць паміж загалоўкамі раздзелаў
    \renewcommand{\cftbeforesecskip}{0em}
    % адлегласць паміж нумарам падраздзела і яго назвай
    \renewcommand{\cftsubsecnumwidth}{20pt}
    % слова ЗМЕСТ пішацца вялікімі літарамі, паўтоўстым шрыфтам, па цэнтру
    \renewcommand{\cfttoctitlefont}{
        \hfill
        \fontsize{14}{14}
        \bfseries
        \MakeUppercase
    }
    \renewcommand{\cftaftertoctitle}{\hfill}

%%% Падпункт 2.2.6 — змест уключае загалоўкі ўсіх частак тлумачальнай запіскі,
%%%                — у тым ліку раздзелаў, падраздзелаў, дадаткаў
\setcounter{tocdepth}{2} % змест уключае толькі section і subsection


%%% Пункт 3.2 — спіс/пералік
%%%           — дадатак Р
%%%           — калі ў любым пункце ёсць перанос на
%%%           — наступны радок, то гэтыя радкі
%%%           — без водступу абзаца
\usepackage{enumitem}
%%% Адлегласці для 1, 2, 3 узроўня спісаў,
%%% каб значок пераліку быў на адлегласці водступу абзаца
\newlength{\indentForFirstItems}
\setlength{\indentForFirstItems}{\parindent + 2ex}
\newlength{\indentForSecondItems}
\setlength{\indentForSecondItems}{\indentForFirstItems + \parindent}
\newlength{\indentForThirdItems}
\setlength{\indentForThirdItems}{\indentForSecondItems + \parindent}
%%% дабаўляем пералік у выглядзе літар беларускага алфавіта
%%% https://tex.stackexchange.com/questions/211386/use-letters-and-numbers-in-numerate-simultaneously
    \makeatletter
        \AddEnumerateCounter{\asbuk}{\belarusian@alph}{}
    \makeatother
%%% Агульныя параметры для кожнага ўзроўню спіса
    \setlist{
        noitemsep,                          % адсутнасць прамежку паміж радкамі
        nosep,                              % адсутнасць прамежку паміж тэкстам і спісам
        leftmargin=0pt,                     % адсутнасць водступу леваруч
        labelsep=0.55ex                     % адлегласць паміж значком і першай літарай
    }
%%% Пункт 3.2 — перад кожнай пазіцыяй ставіць працяжнік
%%%           — водступ абзаца
    \setlist[1]{
        label=--,                           % значок пераліку
        itemindent=\indentForFirstItems,    % водступ абзаца
    }
%%% Пункт 3.2 — пералік у адной з пазіцый,
%%%           — то ставіць малую літару з дужкай
%%%           — водступ абзаца
    \setlist[2]{
        label={\asbuk*}),
        itemindent=\indentForSecondItems
}
%%% Пункт 3.2 — наступны пералік,
%%%           — пры дапамозе арабскіх літар з дужкай
%%%           — водступ абзаца
    \setlist[3]{
        label={\arabic*}),
        itemindent=\indentForThirdItems
    }

%%% Табліцы
\usepackage{array,tabularx,tabulary,booktabs} % дадатковыя магчымасці для табліц
\usepackage{longtable}                        % для табліц на некалькі старонак
\usepackage{multirow}                         % зліццё радкоў у табліцы
\usepackage{adjustbox}                        % для афармлення блокаў з пэўным выраўноўвання

%%% Пункт 3.1 — нумар табліцы складаецца з нумару раздзела і
%%%           — парадкавага нумару табліцы, падзеленых кропкай
\renewcommand{\thetable}{\thesection.\arabic{table}}

%%% Пункт 3.1 — раздзел мае пачынаецца з новай старонкі
\let\oldsection\section
\renewcommand{\section}{\newpage\oldsection}

%%% Пункт 3.1 — табліца/ілюстрацыя мае адасабляцца пустым радком зверху і знізу
\setlength\intextsep{1.3\baselineskip}


%%% малюнкі
\usepackage{graphicx}
    \graphicspath{{images/}} % месца, дзе шукаць графічныя файлы

%%% Пункт 3.1 — нумар табліцы складаецца з нумару раздзела і
%%%           — парадкавага нумару табліцы, падзеленых кропкай
\renewcommand{\thefigure}{\thesection.\arabic{figure}}
